\input texbase

\titulo{Exercício Programa 1}
\materia{MAC0463 - Programação Móvel}

\aluno{Diogo Haruki Kykuta}{6879613}
\aluno{Fernando Omar Aluani}{6797226}

\begin{document}
\cabecalho

\section{Motivação}
Bem, a motivação foi bem simples... Esse é um trabalho que vale nota desta matéria.
Quanto à escolha do feed da Agência USP de Notícias, nós escolhemos ele aleatoriamente
dentre as possíveis opções.


%%%%%%%%%%%%%%%%%
%como usar o programa, cada funcionalidade...
\section{Manual}
A interface do programa é bem simples e fácil de usar, não muito diferente de \textit{rss readers}
existentes para plataformas móveis.

O programa começa com uma \textit{splash-screen} mostrando o carregamento inicial, e depois
mostra a tela inicial de listagem das entradas do feed.

\begin{itemize}
    \item \textbt{Página de Listagem}: lista as entradas do feed da categoria seleciona (mostrada no cabeçalho). Funcionalidades:
        \begin{itemize}
            \item \textbt{Atualizar}: apaga o cache dos feeds dessa categoria, e puxa os feeds do site novamente.
            \item \textbt{Config}: vai para a página de configurações.
            \item \textbt{ALL}: muda a categoria selecionada para <ALL> (todas categorias).
            \item \textbt{Outras Categorias}: vai para o menu de seleção de categorias.
            \item Clicando em qualquer item do feed vai para a página de notícia.
        \end{itemize}
    \item \textbt{Página de Seleção de Categorias}: mostra as categorias do feed.
        \begin{itemize}
            \item \textbt{Voltar}: volta para a página anterior.
            \item \textbt{Config}: vai para a página de configurações.
            \item Clicando em qualquer categoria vai para a página de listagem para a categoria selecionada.
        \end{itemize}
    \item \textbt{Página de Notícia}: mostra em detalhes uma notícia selecionada (uma entrada do feed - um dos items na página de listagem).
    \item \textbt{Página de Configuração}: possibilita a consulta e modificação dos valores de configuração do app.
        \begin{itemize}
            \item \textbt{Voltar}: volta para a página anterior.
            \item \textbt{Cancelar}: cancela as alterações feitas.
            \item \textbt{Salvar}: salva as alterações feitas.
        \end{itemize}
\end{itemize}

%%%%%%%%%%%%%%
%breve explicação do funcionamento da implementação de cada parte significativa do app
\section{Estrutura do Código}
Partindo de um template de projeto padrão do PhoneGap, só alteramos coisas na pasta 
\textit{assets/www}. Criamos outro arquivo .css, alteramos o index.html e mexemos bastante
nos arquivos .js na pasta \textit{js}.

Implementamos a maior parte do código javascript no arquivo \textit{js/index.js}.
Ele define um objeto \textit{app} com as funções e atributos usados pelo programa.

Para adquirir o feed, usamos a Google Feed API. O app carrega ela dinâmicamente pela web,
usando a própria API do Google para tal.
Essa API não só deixa o trabalho de pegar o feed bem fácil como ela também permite pegar mais
entradas do que o site normalmente manda, usufruindo do cache do feed existente nos servidores
do Google.

Em relação à interface gráfica com o usuário, usamos a biblioteca jQuery Mobile, que é multi
plataforma e ajuda com a criação da interface.

A partir que gerencia as configurações e o cache do feed é o sub-objeto \textit{app.settings}.
Ele define as funções que cuidam dos valores de config e do cache. Para guardar os dados de 
config e o cache de forma persistente no dispositivo usamos em ambos os casos o sistema
de \textit{localStorage} do HTML5, quer permite salvar dados de forma persistente no formato
de string numa estrutura que funciona como uma hashtable.


%%%%%%%%%%%%%
%sobre o processo de aprendizado, desenvolvimento e testes do app
\section{Conclusões}
Nós concluimos que fazer uma aplicação multi-plataforma para \textit{mobiles} não é muito fácil,
mas existem ferramentas que ajudam bastante com isso.

Sendo acostumados a programar em linguagens de alto nível como C++, Java e Python, fazer o app 
inteiramente em HTML e JavaScript foi bem confuso no início. Perdemos muito tempo no começo
descobrindo como as coisas funcionavam pra conseguir alterá-las ou implementar outras.
Depois que entendemos como funcionava a estrutura do app para o PhoneGap, ai começamos a 
trabalhar mais eficientemente.

Outra coisa que ajudou bastante no desenvolvimento e nos testes foi o Android Developer Tools,
o qual usamos pra desenvolver o app e testá-lo. Uma vez que descobrimos como gerar mensagens de
log no app e vê-las no ADT conseguimos \textit{debuggar} alguns problemas que não conseguíamos 
antes. E a partir do ADT foi bem fácil testar o app diretamente em nossos dispositivos Android.

\end{document}
